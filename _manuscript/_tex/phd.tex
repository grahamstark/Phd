% Options for packages loaded elsewhere
\PassOptionsToPackage{unicode}{hyperref}
\PassOptionsToPackage{hyphens}{url}
\PassOptionsToPackage{dvipsnames,svgnames,x11names}{xcolor}
%
\documentclass[
  letterpaper,
  DIV=11,
  numbers=noendperiod]{scrartcl}

\usepackage{amsmath,amssymb}
\usepackage{iftex}
\ifPDFTeX
  \usepackage[T1]{fontenc}
  \usepackage[utf8]{inputenc}
  \usepackage{textcomp} % provide euro and other symbols
\else % if luatex or xetex
  \usepackage{unicode-math}
  \defaultfontfeatures{Scale=MatchLowercase}
  \defaultfontfeatures[\rmfamily]{Ligatures=TeX,Scale=1}
\fi
\usepackage{lmodern}
\ifPDFTeX\else  
    % xetex/luatex font selection
\fi
% Use upquote if available, for straight quotes in verbatim environments
\IfFileExists{upquote.sty}{\usepackage{upquote}}{}
\IfFileExists{microtype.sty}{% use microtype if available
  \usepackage[]{microtype}
  \UseMicrotypeSet[protrusion]{basicmath} % disable protrusion for tt fonts
}{}
\makeatletter
\@ifundefined{KOMAClassName}{% if non-KOMA class
  \IfFileExists{parskip.sty}{%
    \usepackage{parskip}
  }{% else
    \setlength{\parindent}{0pt}
    \setlength{\parskip}{6pt plus 2pt minus 1pt}}
}{% if KOMA class
  \KOMAoptions{parskip=half}}
\makeatother
\usepackage{xcolor}
\setlength{\emergencystretch}{3em} % prevent overfull lines
\setcounter{secnumdepth}{-\maxdimen} % remove section numbering
% Make \paragraph and \subparagraph free-standing
\makeatletter
\ifx\paragraph\undefined\else
  \let\oldparagraph\paragraph
  \renewcommand{\paragraph}{
    \@ifstar
      \xxxParagraphStar
      \xxxParagraphNoStar
  }
  \newcommand{\xxxParagraphStar}[1]{\oldparagraph*{#1}\mbox{}}
  \newcommand{\xxxParagraphNoStar}[1]{\oldparagraph{#1}\mbox{}}
\fi
\ifx\subparagraph\undefined\else
  \let\oldsubparagraph\subparagraph
  \renewcommand{\subparagraph}{
    \@ifstar
      \xxxSubParagraphStar
      \xxxSubParagraphNoStar
  }
  \newcommand{\xxxSubParagraphStar}[1]{\oldsubparagraph*{#1}\mbox{}}
  \newcommand{\xxxSubParagraphNoStar}[1]{\oldsubparagraph{#1}\mbox{}}
\fi
\makeatother


\providecommand{\tightlist}{%
  \setlength{\itemsep}{0pt}\setlength{\parskip}{0pt}}\usepackage{longtable,booktabs,array}
\usepackage{calc} % for calculating minipage widths
% Correct order of tables after \paragraph or \subparagraph
\usepackage{etoolbox}
\makeatletter
\patchcmd\longtable{\par}{\if@noskipsec\mbox{}\fi\par}{}{}
\makeatother
% Allow footnotes in longtable head/foot
\IfFileExists{footnotehyper.sty}{\usepackage{footnotehyper}}{\usepackage{footnote}}
\makesavenoteenv{longtable}
\usepackage{graphicx}
\makeatletter
\def\maxwidth{\ifdim\Gin@nat@width>\linewidth\linewidth\else\Gin@nat@width\fi}
\def\maxheight{\ifdim\Gin@nat@height>\textheight\textheight\else\Gin@nat@height\fi}
\makeatother
% Scale images if necessary, so that they will not overflow the page
% margins by default, and it is still possible to overwrite the defaults
% using explicit options in \includegraphics[width, height, ...]{}
\setkeys{Gin}{width=\maxwidth,height=\maxheight,keepaspectratio}
% Set default figure placement to htbp
\makeatletter
\def\fps@figure{htbp}
\makeatother

\KOMAoption{captions}{tableheading}
\makeatletter
\@ifpackageloaded{caption}{}{\usepackage{caption}}
\AtBeginDocument{%
\ifdefined\contentsname
  \renewcommand*\contentsname{Table of contents}
\else
  \newcommand\contentsname{Table of contents}
\fi
\ifdefined\listfigurename
  \renewcommand*\listfigurename{List of Figures}
\else
  \newcommand\listfigurename{List of Figures}
\fi
\ifdefined\listtablename
  \renewcommand*\listtablename{List of Tables}
\else
  \newcommand\listtablename{List of Tables}
\fi
\ifdefined\figurename
  \renewcommand*\figurename{Figure}
\else
  \newcommand\figurename{Figure}
\fi
\ifdefined\tablename
  \renewcommand*\tablename{Table}
\else
  \newcommand\tablename{Table}
\fi
}
\@ifpackageloaded{float}{}{\usepackage{float}}
\floatstyle{ruled}
\@ifundefined{c@chapter}{\newfloat{codelisting}{h}{lop}}{\newfloat{codelisting}{h}{lop}[chapter]}
\floatname{codelisting}{Listing}
\newcommand*\listoflistings{\listof{codelisting}{List of Listings}}
\makeatother
\makeatletter
\makeatother
\makeatletter
\@ifpackageloaded{caption}{}{\usepackage{caption}}
\@ifpackageloaded{subcaption}{}{\usepackage{subcaption}}
\makeatother
\ifLuaTeX
  \usepackage{selnolig}  % disable illegal ligatures
\fi
\usepackage{bookmark}

\IfFileExists{xurl.sty}{\usepackage{xurl}}{} % add URL line breaks if available
\urlstyle{same} % disable monospaced font for URLs
\hypersetup{
  pdftitle={Graham Stark PHd},
  pdfauthor={Graham Stark},
  pdfkeywords={Economics, thing},
  colorlinks=true,
  linkcolor={blue},
  filecolor={Maroon},
  citecolor={Blue},
  urlcolor={Blue},
  pdfcreator={LaTeX via pandoc}}

\title{Graham Stark PHd}
\author{Graham Stark}
\date{}

\begin{document}
\maketitle
\begin{abstract}
this is the abstract.
\end{abstract}

Understanding the distributive impacts of tax-benefit policy:
development of microsimulation techniques to provide new insights into
reform

PhD by Published Works, Graham Stark

Abstract

Introduction

My PhD thesis builds on decades of academic research conducted as Senior
Research Officer at the Institute for Fiscal Studies (IFS), the Open
University and as Director of Virtual Worlds. I present three
publications developed in my role as Senior Research Policy at
Northumbria which emerges from this work. I am a leading specialist in
microsimulation modelling of tax-benefit systems, programming software,
databases and interfaces. Most significantly, I am the creator of TAXBEN
(Johnson, Webb \& Stark 1990), the IFS' tax and benefit microsimulation
model (Giles \& McCrae 1995), which continues to be used to analyse
Government tax-benefit reform.

Having started my career in Economics at Lancaster University, I spent
18 years as Senior Research Officer at the IFS, before starting my own
consultancy, Virtual Worlds. During my time at the IFS, I produced a
body peer-reviewed journal articles (listed) and reports that achieved
impact with policy makers and Government departments, such as the
DSS/DWP and provided rapid response assessment of Budgets for media
outlets. In this role, I published highly impactful research with
colleagues such as Paul Johnson, Director of the IFS, Andrew Dilnot,
Warden of Nuffield College, Oxford, and former Director of the Institute
for Fiscal Studies (1991-2002), Principal of St Hugh's College, Oxford
(2002-12), and Chair of the UK Statistics Authority (2012-17) and Evan
Davis (e.g.~Davis et al.~1992).

After founding Virtual Worlds, an economic microsimulation consultation,
I expanded my body of work to examine such issues as tax, social care
and legal aid reform at UK and devolved national (England, Scotland,
Wales and Northern Ireland) level. I have also provided a range of
analyses on tax-benefit reform in other countries, including in the US
and low-middle income societies in Africa and Europe. Importantly, I
have created a series of online tools that enable users to analyse
microsimulation data in accessible form.

Since 2022, I have been Senior Research Associate in Public Policy at
Northumbria University, providing economic and microsimulation insight
that has fed into a series of peer-reviewed publications, including two
that are in press. In this role, I have developed the TriplePC (Public
Policy Preference Calculator, https://triplepc.northumbria.ac.uk/), a
unique online microsimulation tool that enables projection of the
health, health economic, economic and public opinion impacts of
customisable welfare policies, which is described in a forthcoming
article in the International Journal of Microsimulation (Stark et
al.~2024).

This thesis summary examines the underpinning research from three recent
publications that assess the impact post-Financial Crisis era
interventions and potential interventions. The first report (Reed \&
Stark 2018) forecasts the impact of the Scottish Government's Child
Poverty Plan. The second (Reed \& Stark 2020) assess the impact of
policy for care leavers in England. The final report, which summarises
the research conducted for Stark et al.~(2024), describes development of
the TriplePC as part of a broader NIHR-funded tranche of research on the
impact of basic income on poverty, inequality and health.

Chapter 1: Microsimulation modelling

Microsimulation Modelling

(Johnson and Stark 19, King and Stark 19xx, Coulter, Heady and Stark
199x, King and Stark 1985, Duncan and Stark 20xx)

Microsimulation is the study of social phenomena using computer
simulations on individual-level data, such as people and companies.

The canonical microsimulation is the Tax Benefit Model, which models the
effects of the fiscal system on individuals and households. As such, I
developed TAXBEN2, the complete model rewrite. An updated version of
that that model remains in use today. 25 years on it still has many
technical advantages over competitor models such as Euromod. The
Institute for Fiscal Studies had a such a model before I joined, but it
was unsatisfactory in many ways. The application of TAXBEN2 included
analysis of the UK Capital Gains Tax (King and Stark), and calculation
of the often highly non-linear relationships between what people earn
and what they get to keep (Duncan \& Stark 2000). Subsequently, we built
similar models for former Soviet Block countries using similar methods.
Coulter, Stark and Smith (1995) describes one such model for the former
Czechoslovakia. King and Stark (?)

Applications of Microsimulation

(Stark 198x, Dilnot and Stark 19xx 19xx, Johnson and Stark 19xx, 19xx)

The most impactful aspect of microsimulation analysis lies in
forecasting important possible policy changes. Stark (1986) analyses the
move from taxing husbands and wives jointly to the individual taxation
we have today. One consequence of that change was that families with two
earners could now receive more tax allowances than families with a
single earners, and this paper is one of several I wrote exploring
proposed `transferrable allowances' that would address this. As with all
the papers I produced during this period, it is possible that some of
this analysis seems commonplace now. However, the ability to describe
the impact of such changes on the incomes of different types of family,
incentives to work, and tax revenues was very new at the time and
attracted widespread interest among the media, policy makers, and
academics.

For example, Dilnot and Stark (1986a; 1986b) are analyses of the now
widely-recognised poverty trap. As a family's income rose from a low
level, the withdrawal of means-tested benefits along with increases in
taxes could leave them no better off, or even worse off. These papers
were the first to show that the numbers of families affected by very
high withdrawal rates were at the time likely quite small. Taper rates
in Universal Credit and the general increase in the number of people on
low incomes means that that number is much higher, emphasising that the
relevance of the work some 38 years on.

Likewise, Johnson and Stark (1991) was the first study of the
distributional effects of a UK minimum wage. It showed that the likely
gains were mostly amongst second earners, meaning that the gains were
predominantly to the middle of the income distribution. This result was
not appreciated at the time, but has largely been overtaken by social
changes. Johnson and Stark (199) was my attempt at a `How to Lie with
statistics' paper specifically about tax policy, and written as a
reaction to media coverage of the tax policies proposed in the 1992
general election. Stark and Johnson (1989) summarised the entirety of
the tax and benefit policy of the Thatcher Government in a concise but
consistent way, showing how overall changes were highly regressive. This
work, overall, has highlighted the extent to which the 1979-1997
Governments produced significant increases in inequality and presented
analysis of means of achieving progressive reform.

Benefit Takeup

(Fry and Stark 198x, 1994; Buck and Stark 20xx, 20yy))

The UK's benefit system is largely means-tested. Entitlement to
Universal Credit, Working Tax Credit and the like depend on family
income, with benefits being withdrawn as income rise. Along with the
Poverty Trap discussed above, the key problem with means testing is that
these benefits may not be claimed, largely because of stigma or because
of the complexity of claiming them. Ours were the first studies to use
microsimulation techniques to study non-take up (e.g.~Fry \& Stark 1987;
1992; 1993). A key result is that takeup is higher for large
entitlements, which provides support for the `disutility' model of
take-up. Studies using our microsimulation methods have since become a
mini-industry (Moffat xx, (some survey)).

Highlighting complexity and stigma as likely important factors in
limiting the effectiveness of means-tested benefits has clear relevance
to my later work on less conditional welfare reforms, such as basic
income, but my initial response was to examine more focused policies
such as Legal Aid Means-Testing. I used microsimulation to answer the
question: `what is the simplest set of rules that could achieve some set
of objectives - the numbers and types of families eligible, overall
expenditure and administrative costs and so on?' The work had
significant impact, directly contributing to reforms to Legal Aid means
tests, though the positive impact was largely cancelled out by
subsequent large cuts to the legal aid budget. Subsequently, there was
interest in the simplification question in Government, including an
`Office for Tax Simplification'. However, this often combined
simplification with sweeping distributional changes that produced more
complex implementation issues.

Budget Analysis

One of the key founding missions of IFS was to provide timely, detailed
analysis of actual and proposed budget changes. My work with Paul
Johnson, Andrew Dilnot, Evan Davis and others illustrate the
contribution of microsimulation to this (Johnson \& Stark 1991; Davis et
al.~1992; Johnson, Webb \& Stark 1987 Dilnot, et al.~1987). These, too,
may seem commonplace now, although analysis of budgets is mostly
focussed on public finances in the aggregate. However, at the time,
rapid, detailed analysis of the distributional and incentive effects of
budgets was new and had a huge impact in the wider discourse.

My more recent work in this area includes papers for the Office of the
Scottish Charities Register (OSCR), which were concerned with granting
charitable status to institutions, such as private schools or golf
clubs, that few people could afford (Reed \& Stark 2009). I first
examined the literature on affordability as applied to, for example,
affordable housing, fines levied by the courts and fuel poverty. I then
described a microsimulation model that shows the proportions of
households who might be able to afford the proposed fees of some
applicant for charitable status. Again, I believe this was the first
model of its kind.

Chapter 2: Austerity era interventions

Chapter 3: Microsimulation of policies for which there is no precedent:
basic income

Prospective welfare policies have often been assessed on their financial
impacts, for example, their effects on net household incomes and
marginal and average tax rates. However, welfare policies can also have
a substantial effect on population health and wellbeing. In addition,
politicians must consider the electoral implications of policies that
would affect large sections of the population. In my third report,
Treating Causes not Symptoms (Johnson et al.~2023), I describe the
Public Policy Preference Calculator (TriplePC), a new microsimulation
model that seeks to extend the microsimulation in two ways.

First, as well as modelling the outcomes of a policy in the conventional
way, our model uses Conjoint Analysis of public acceptability data to
give an indication of the policy's popularity. This is novel and
important. There are measures that might actually be popular with , but
which policymakers have been unwilling to touch because of uncertainty
about their electoral consequences. Perhaps the best example from UK
history is the SDP/Liberal Democrat's `Dead Parrot' merger manifesto of
January 1988 (Gourley, n.d.; Crewe \& King, 1995), which proposed the
abolition of Child Benefit and the imposition of a uniform rate of Value
Added Tax (VAT) to raise money for an anti-poverty program. Although
this had been modelled in detail, fear of the electoral consequences
among Members of Parliament meant the manifesto was abandoned within a
day. The resulting confusion and indecision arguably caused long-lasting
damage to centrist politics in the UK (Crewe \& King, 1995). The UK's
zero-rating for food and children's clothing remains politically
untouchable to this day despite the orthodox economic arguments in
favour of a uniform rate (Crawford et al., 2010). But would VAT
extension really be unpopular, especially if it was part of a package
that used the money raised for poverty reduction or other appealing
policies? Our approach allows us to address questions like this.

Second, I integrated health outcomes into the model. There is strong
evidence that welfare policies can have a substantial effect on
population health (Johnson et al., 2022). A stark reminder of the real
impact of worsening population health can be seen in the proportion of
the UK population with a long-standing illness, disability or impairment
which causes substantial difficulty with day-to-day activities. This is
estimated to have risen from 19\% in 2011/12 to 24\% in 2021/22, an
increase of 3.9 million people (Department for Work and Pensions, 2023).
Indeed, the estimate increased from 14.1 million in 2019/20 to 16.0
million in 2021/22 (Department for Work and Pensions, 2023).
Interestingly, the proportion among state pension age adults has
remained the same between 2011/12 and 2021/22 at 45\%, whereas for
working-age adults it has increased from 16\% to 23\% and for children
the figures are 6\% to 11\%. This suggests that increases in prevalence
are not simply the effect of an ageing population (Department for Work
and Pensions, 2023).

In that context, it is essential that policymakers invest real thought
in realising the Government's prevention agenda (Department of Health
and Social Care, 2018), which was incorporated into the 2019 NHS England
Long Term Plan (NHS England, 2019). Forty-three years on from the Black
Report which highlighted the role of material circumstances on health
inequalities (Working Group on Inequalities in Health, 1980), 13 since
the Marmot Review (Marmot et al., 2010) and three since its 10-years-on
update (Marmot et al., 2020) which highlighted worsening trends in
inequalities, there is good reason to examine and tackle social
determinants of health.

Some of the authors of this article (Johnson et al., 2022) have called
for trials of cash transfers, in particular Basic Income, as an upstream
intervention to mitigate poverty, inequality and insecurity as social
determinants of mental and physical ill-health. Systematic reviews of
cash transfer schemes that resemble Basic Income, such as Gibson et
al.~(2020), have indicated positive impacts on mental and physical
health, hospital attendance and health related behaviour, such as
alcohol and drug use. In contrast, conditional, means- and needs-based
welfare systems in high-income countries are associated with below
average health outcomes (Shahidi et al., 2019) and increased
psychological distress prevalence (Wickham et al., 2020). We have
suggested several explanations (Johnson et al., 2022): schemes are
`insufficient to offset the negative health consequences of severe
socioeconomic disadvantage' (Shahidi et al., 2019); conditionality and
assessment inflicts stress (Dwyer et al., 2020) and creates perverse
incentives for health-diminishing behaviour (Johnson et al., 2022); and
focusing on the poorest fails to mitigate broader determinants that
affect society as a whole (Marmot et al., 2010).

There are existing health simulations for the UK. Public Health
Scotland's Informing Interventions to reduce health Inequalities (Triple
I) tools focus on comparing the effects and costs of a range of
tax-benefit (including a Basic Income) changes as well as non-economic
programmes such as a lifestyle weight management service,
20-mile-per-hour speed limits or Alcohol Brief Interventions. Effects
and costs modelled are based on premature deaths, years of life lost and
hospital stays, with changes population health and inequalities as the
key measures. It does not model the economic impacts on household types
nor public preferences.

Most other health microsimulations tend to model the economic effects of
health, rather than the other way round (Schofield et al., 2017). There
are some that model the effects of, potentially economic, interventions
on health, particularly with regard to extrapolating from childhood and
adolescence, for example, the University of York's LifeSim (Skarda et
al., 2021).

I therefore decided to create microsimulation with a public-facing user
interface -- the Public Policy Preference Calculator (TriplePC) -- that
would enable automated assessment of economic and health impacts as well
as public preferences between different welfare and tax policies. The
TriplePC model project therefore had three strands:

\begin{verbatim}
Estimating the likely electoral popularity of possible policies 

Deriving relationships between income and health, in a form suitable for use in a microsimulation 

The integration of strands 1 and 2 into a microsimulation tax-benefit model 
\end{verbatim}

We discuss these in turn.

Public preferences: Conjoint analysis

Conjoint analysis (Hainmueller et al., 2013) is a survey-based technique
originally developed as a market research tool, to examine how consumers
value characteristics (sweetness, colour, alcohol content, etc.) of
goods. Recently, the technique has become popular as a method for
discovering the public's relative valuations of competing economic or
social policies (Bremer \& Bürgisser, 2023). Research comparing conjoint
survey experiments to actual votes has shown that the conjoint results
are good predictors of voting outcomes (Bansak et al., 2023).

Our study is described in full in (Nettle et al., 2023), with the data
available as (Johnson et al., 2023). The authors of that study recruited
800 UK resident adults through the Prolific online platform.
Participants were asked repeatedly to choose their preferred welfare
policy from sets of two. Each time, the two policies contained the same
input (design) and outcome (health and distributional) attributes but
with randomised levels in each (e.g.~payment sizes of £63 per child,
£145 per adult and £190 per pensioner; poverty decreased by 25\% etc.).
Each participant completed 15 choice tasks. Each option within each task
was defined by 10 attributes. Each attribute had three to nine possible
levels. Table 1 shows the full list of 10 attributes with between three
and nine levels each.

Relative to the UK population, the sample contained an
over-representation of people who

voted for the left to centre-left Labour party at the 2019 general
election (44.3\% of those in

our sample who voted, vs.~32.1\% election result); and an
under-representation of those

who voted for the right to centre-right Conservative party (31.8\%
versus 43.6\% election

result). Although we did create survey weights to correct for this,
TriplePC currently uses unweighted data.

All options were fully randomly generated from the possible
combinations. Instructions in the survey explained that participants
might prefer some features in one policy and some in the other, but they
needed to consider which policy they preferred overall. The attributes
on which the policies varied were explained in greater depth prior to
the first choice task, and then described just with brief phrases during
the choice tasks themselves. Note that many of these randomly generated
pairs are totally implausible. This is integral to how a fully
randomized conjoint analysis works: by generating all the possible
combinations, things that are actually correlated in real life (e.g.~tax
rates and poverty rates) are rendered orthogonal to one another. It is
this that allows identification of their independent marginal effects.

We simultaneously estimated the average impact of preference or
dis-preference for particular feature attribute-value on preference for
policies overall using comparable scales through computation of Average
Marginal Component Effects (AMCEs) (Hainmueller et al., 2013) from
linear probability models. The AMCE for a given level of an attribute
can be interpreted as the marginal effect on the probability of choice
of the attribute being at that level compared to the reference level,
averaging across the possible levels of all other attributes. Through
randomization and a high number of pairwise comparisons, this allows us
to quantify the causal effect of including specific levels of individual
reform elements on the support for the entire reform package, compared
with the support for a reform package that contains the baseline level
(status quo) of this particular reform element (Nettle et al., 2023a).
That study found:

\begin{verbatim}
preference for more generous payments than less generous ones. 

strong preference for decreases in poverty (compared to the status quo). 

preferences on tax rates depended on the broad effects of the policy package. Increasing personal income tax rates were popular if the package they were part of also decreased poverty, and unpopular otherwise. 

preference for a wealth tax, carbon taxes, and increased corporation tax, as opposed to increased government borrowing. 

significant positive effect of a large reduction in inequality, and a significant negative effect of a large increase in inequality. However, the effects for inequality were weaker than for poverty. 

other health and wellbeing consequences also had some significant marginal effects above and beyond those of poverty and inequality. An increase in life expectancy of five years was significantly preferred to the status quo, and a decrease in life expectancy of five years significantly dis-preferred. 

dis-preference for increased rates of anxiety and depression relative to the status quo, and there was a slight preference for policies that decreased them sharply. 

no strong preference for or against means-testing or other restrictions on eligibility. 

Mild differences between left- and right- supporting participants, in the expected directions. 

no significant variation by gender or between rich and poor. 

older people were significantly less keen on high income tax schemes and, curiously, less concerned with heath consequences. 
\end{verbatim}

For the TriplePC, we extract the AMCEs for each component of any welfare
policy the user wishes to specify through the interface from the data of
Nettle et al.~(2023a, b). In the conjoint methodology, these are assumed
to have an additive effect on popularity. That is, for example, the
popularity of a policy combining high income taxes, carbon tax, and a
decrease in poverty will be the sum of the negative marginal effect on
popularity of the high income tax, the positive marginal effect on
popularity of the carbon tax, the positive marginal effect of the
poverty reduction, and so on. This summation is the popularity outcome
returned by the model.

The conjoint analysis was conducted ahead of the construction of the
microsimulation model. As we discuss below, some of the measures in
Table 1 (income tax, payment sizes, poverty and inequality rates) are
reasonably straightforward to model (though there are issues around
definitions). Others, such as the `other funding' options and the
relationship between income and health, are harder.

Modelling health outcomes

We model two health measures: mental health and life expectancy. We
build a model relating SF-12 scores (Ware, 2002) to income and
demographic characteristics. SF-12 is a widely used measure of an
individual's health-related quality of life, with two summary scores:
the Physical Component Summary (PCS-12) and the Mental Component Summary
(MCS-12). The model is estimated over 12 waves (2009/11-2020/22) of
Understanding Society: The UK Household Longitudinal Study (UKHLS)
(Institute for Social and Economic Research, 2023) panel data (Reed et
al., 2024).1 Another companion article (Reed et al., 2023a) discusses
this modelling in detail.

Health modelling strategy

Our health model is estimated using the `between' individual coefficient
from a fixed effects `within-between' model, which combines the effect
on physical and mental health of both an individual's income in one wave
vs their average across waves, and their average across waves compared
with the sample average.

The model is a reformulation of the standard Mundlak model and has a
significant advantage in being able to retain the flexibility of random
effects models while reducing concerns about bias that fixed effects
models address (Bell et al., 2019; Bell \& Jones, 2015). The
within-between model, conceptually, captures several key income-based
drivers of health, including:

\begin{verbatim}
temporary income shocks (within component), which see individuals’ income increase or decrease in one wave compared to their average. 

permanent income shocks (between component), which see an individual’s average income either be closer to or further away from the population average. 

objective inequality (between component), which see differences between individuals’ average income, which is calculated over a longer, enduring, period. 

subjective social status inequality (between component), which is the psychological phenomenon driven, in part, by income inequality. 
\end{verbatim}

It does not, however, capture what we anticipate through our model of
impact (Johnson et al., 2022) to be very substantial benefits from
systems such as Basic Income of increased security of income and
protection from destitution for a very large proportion of the
population in even relatively highly paid jobs. We use the
between-individual coefficient in our modelling because changes to the
welfare system are more likely to reflect permanent income shocks. The
SF-12 regression are available in a working paper (Reed et al., 2024).

We create a binary variable for cases of depressive disorder which takes
the value of 1 if the individual's imputed MCS-12 score is ≤45.6 and 0
otherwise (Vilagut et al., 2013). We impute life expectancy from SF-12
in three steps:

\begin{verbatim}
Convert SF-12 scores to SF-6D (Brazier et al., 2002), using software from QualityMetric (QualityMetric, 2022). SF-6D is a preference-based measure of health. 

Use SF-6D score to calculate quality-adjusted life years (QALYs). QALYs are a widely recognized standardised measure of health outcomes commonly used in health economics (Drummond et al., 2015; Kaplan & Hays, 2022). 

Calculate life expectancy from QALYs using multipliers conditional on gender and age. The multipliers are derived from McNamara et al. (2023). 
\end{verbatim}

I made a series of assumptions regarding the responsiveness of change to
income. These estimates are then integrated directly into the
microsimulation model. In figure 1) below, the SF-12 Mental Health
outputs are shown in the bottom left row. As modelled, mental health
appears rather unresponsive to changes in income. For example, choosing
the most generous benefit increases from the options from the Conjoint
Analysis -- a package costing £430bn -- would reduce the number of
adults with critically low SF-12 scores (below 45.6) by about 3\%.

Microsimulation: The TriplePC model

In conjoint surveys, respondents have preferences over inputs (income
tax rates, payment sizes, etc.) and outcomes (poverty and inequality
levels, numbers of mental health cases, etc.). We use microsimulation to
bridge between them

The analysis uses a heavily adapted version of Scotben (Stark 2023a), an
open-source microsimulation model of Scotland written in the Julia
programming language. Scotben is a conventionally structured static
tax-benefit model, in the family of models branching out from the
Institute for Fiscal Studies' TAXBEN (Johnson et al., 1990) of which two
of this article's authors (Reed and Stark) were developers. For this
project, we extended the scope of the model to Great Britain2 using a
single 2021/22 wave of Family Resources Survey (FRS) data (Department
for Work and Pensions, 2019). The model covers the whole of the UK
personal tax and benefit system, including local taxes, with some minor
exceptions such as Scottish Best Starts grants. As a base, we use the
tax and benefit system as of Q3 2023, and all incomes, wealth and
consumption are also uprated to 2023 Q3. We use the 2021/22 FRS sample
weights as-is.

The outcome questions are phrased as changes (e.g.~`50\% fewer cases of
anxiety and depression', `Poverty increased by 50\%'). A particularly
tricky question arising from this is how to establish a baseline for
comparison. The conjoint experiment survey had no `keep things as they
are' option for the tax and benefit inputs, so there were two options
for the TriplePC:

\begin{verbatim}
Using a tax-benefit system some way from the current one as baseline and assuming that the outcome changes represent changes in poverty, health, etc. from that point, rather than changes from the actual current situation. 

Using the current system as the baseline, but then the default output will have significant deviations for the outcome variables. 
\end{verbatim}

Neither of these choices is good but on balance we decided 1) was the
least bad. I It makes the conjoint popularity output much easier to
understand: with option 2) we would be applying large changes in poverty
and inequality to the base conjoint results which would make those
results very unintuitive. The result, however, is that the model starts
some distance from the actual existing system. This is certainly a
lesson for future work.

Income tax rates

The conjoint experimental survey had six income tax rate options with a
basic, higher and additional rate in each (see Table 1). The first of
these options is the current non-Scottish UK income tax rates, which we
take as the base.3 All other options represent rate increases. We assume
the corresponding thresholds are as present. Since we have to remain
consistent with the conjoint analysis, only the six rate groups in Table
1 are presented to the user, though the model can handle any combination
of rates and thresholds. We assume no behavioural responses to changing
tax rates and make no corrections for under-reporting of incomes beyond
that embodied in the FRS sample weights.

Benefits

The payment size question in the conjoint survey was about a
hypothetical system of payments that most closely reflects the
simplicity of Basic Income (Reed et al., 2023b). There were also
questions about eligibility, means-testing and citizenship (see Table 1,
above). It is not clear how this proposed system of cash transfers
should interact with the existing tax and benefit system, especially
bearing in mind that the question is not how an expert believes they
should interact, but what was most likely in the mind of the conjoint
respondents. We follow our recent analysis (Reed et al., 2023b) and
assume:

\begin{verbatim}
means-tested benefits are retained.4 

most other benefits, including the State Pension and Child Benefit, are abolished and replaced by the cash transfers. 

Needs-based benefits such as those based on sickness or disability, like Personal Independence Payment (PIP), are retained. 
\end{verbatim}

The least generous set of options (Child - £0; Adult - £63; Pensioner -
£190) are taken as the base values. Compared to the actual system, this
means that we're starting from a social security system that's
considerably more expensive (because of the adult payments), but where
pensioners are usually slightly worse off (£190 vs £203.85 for the new
State Pension) and families with large numbers of children not on
means-tested benefits are worse off, since the cash transfer to children
is zero in the default case and the payments to adults are not always
enough to compensate. We do not adjust taxes to meet these extra base
costs. For eligibility, means-testing and citizenship options, it seemed
plausible that at least some of the respondents might be aware of the
means and eligibility tests from existing benefits. Consequently, we
model the eligibility rules that apply to the `legacy' UK benefits:
Working Tax Credit and Income Support/Employment Support that are in the
process of being phased out, and the means-tests are taken from the new
Universal Credit (Child Poverty Action Group, 2022).

Modelling other funding options

Table 1 includes a number of `other funding' options that are worth
discussing briefly.

There is ambiguity in some of these options. A wealth tax or carbon levy
could be implemented in many ways, for instance; we make what we hope
are reasonable assumptions for these cases, but for the microsimulation
to be fully consistent with the conjoint survey we would have to know
what was in the mind of the respondents. This section discusses how we
tackled modelling three of these options.

`Increase in VAT' (Value Added Tax)

Our Family Resources Survey (FRS) dataset has no expenditure data. The
main UK source of household expenditure data is the Living Costs and
Food Survey (LCF) (Office for National Statistics, 2019a). To model the
complex set of VAT exemptions and zero-rated goods5, we therefore have
three choices:

\begin{verbatim}
Switching the primary dataset to be the LCF. LCF6 was the primary dataset of all UK Tax Benefit Models until it was supplanted by the FRS. LCF remains the source used in the Treasury’s IGOTM model (Brice, 2015). But to model other options such as wealth taxes, we would need still other datasets. Switching between multiple different datasets, and hence slightly different base outcomes, depending on which options were being modelled could be confusing. 

Imputing expenditure data onto the FRS via a demand system. This seems appealing as the model would be consistent with economic theory, expenditure could vary with tax rates, and, in principle, we could use the demand system to calculate changes in economic welfare rather than just estimate cash changes. But it is infeasible to build a demand system with fine enough detail to adequately model the complex set of exemptions and zero-rated goods.  

Assigning LCF records to our primary FRS dataset using data matching, which is the option we chose. Since it is important to capture the relationship between income and expenditure, we performed the matching in two steps. We selected a candidate group of LCF donors in the conventional way (matching on age, sex, tenure, etc.) and then ranked among those candidates by income (Stark, 2023). 
\end{verbatim}

In aggregate, our current VAT modelling under-predicts revenues by about
50\%: we model £102bn for VAT revenues as against actual revenues for
2022/3 of £160bn (HMRC 2023). There are well-known problems with
under-reporting LCF expenditure data such as of alcohol and tobacco
spending (Reed 2012) which will account for much of this. For this
exercise, we provide a crude fix by grossing up all the expenditure
records by 1.5. Undoubtedly this is an area we can improve on.

`Tax on wealth'

Modelling wealth is tricky for three reasons:

\begin{verbatim}
Our primary FRS dataset has limited information on wealth, mainly intended to help model benefit eligibility tests. 

The form this wealth tax should take is not specified. 

Wealth taxes are held to be particularly easy to evade or avoid (Scheuer & Slemrod, 2021). 
\end{verbatim}

To solve 1., we impute data from the Wealth and Assets Survey (WAS)
(Office for National Statistics, 2019b) onto the FRS households. We
chose to do this using a simple linear regression of three categories of
wealth (pensions, housing, and financial and other assets) against
household characteristics that are common to FRS and WAS. In retrospect,
regression-based imputation was likely a mistake and matching will be
used in future versions.

For the form of the tax, we were guided by the Wealth Tax Commission
(Advani et al., 2020). We followed their recommendations of excluding
pension wealth, having an allowance of £500,000, and having the tax
payable over five years, though we deviated from the Commission in
applying the tax to aggregate household wealth rather than individual
wealth (Chamberlain, 2020).

Even when payable over five years, as recommended by the Wealth Tax
Commission, the payments from wealth taxes needed to fund some of the
more generous benefit schemes can exceed net income for many families,
especially elderly families who have high housing wealth. Most likely
the scheme would need to be augmented by an income-related rebate
scheme, or some scheme to defer until death.

`Corporation tax increases'

Building a plausible micro-data based model of Corporation Tax is
difficult and not something that could be contemplated for this project.
In any case, for a household-based microsimulation model, what matters
is the incidence of the tax on the households. This could be on profits,
or passed on in price increases or real wage reductions (Harberger,
1962; Atkinson \& Stiglitz, 2015). If we make a simple `small country'
assumption -- that the rate of return on capital and the price of
tradeable goods are set exogenously on world markets -- then Corporation
Tax is ultimately incident on (private sector) wages and self-employment
income. Therefore, we calculate the tax increase needed to meet the
costs of the benefit increase and reduce the wage bill by that amount.
Note that as the wage bill falls, direct tax revenues also fall, but in
a non-linear way because of the tax allowance and progressive tax rate
structures, so finding the correct Corporation Tax increase requires the
use of our root-finder. In practice the rates needed for the more
generous benefit increases can be implausibly large, exceeding in some
cases total UK Corporation's Gross Profits.

Model flow

Putting all this together, a model run has five main stages:

\begin{verbatim}
The user selects from the Payment size, Income tax, Other funding, Conditionality, Means testing, Universality options from Table 1. 

The model then calculates net incomes for each person in the FRS households given these choices. 

These net incomes are in turn plugged in to the equations discussed in Section 2 to give us estimates of changes to the prevalence of depressive disorders and mortality. 

The model next calculates gainers and losers, revenues and costs, and changes in poverty and inequality. 

Finally, the model calculates conjoint public preferences based on 1 to 4 above and displays the results. 
\end{verbatim}

The model has a simple single page web interface, publicly available at
https://triplepc.northumbria.ac.uk/. Figure 1, below, shows this in
action. The user has selected a relatively generous benefit increase
(top left panel), partially paid for by income tax increases (top
centre). The bottom half of the screen shows the results, relative to
the base discussed in section 3 above. Health results are in the bottom
row, showing small mental health improvements. The net cost of this
scheme is £118bn (right middle panel). Poverty and inequality are both
reduced (centre left panel). The Conjoint analysis is in the centre: the
scheme is more popular by 4.7 points than the baseline due to the
popularity of the poverty and inequality reductions and the benefit
increases, though this is partly offset by unpopular tax increases.

A screenshot of a computer screen

Description automatically generated

Figure 1: TriplePC interface example scenario

Reflections on the approach

One important lesson for similar future work is the need for good
coordination between conjoint analysis and microsimulation modelling at
the outset of the project.

In our case, the conjoint analysis was conducted ahead of
microsimulation modelling work. Consequently, microsimulation
requirements were largely fixed by the questions in the conjoint survey.
This has several consequences:

\begin{verbatim}
The model could present only a very limited set of options for taxes and benefits compared to the model’s underlying capabilities. The survey system used for the survey – Qualtrics – had a hard limit on the number of attributes that could be included, which meant that it was not possible to ask about basic and higher tax rates individually. It might also have increased respondent load to an unacceptable level and therefore reduced the quality of the preference data. 

The meaning of options such as ‘Tax on wealth’ should, where possible, be made clearer in order to provide a clear direction for modelling. 

Co-development of a conjoint survey and microsimulation might have enabled respondents to see accurate consequences of their preferred policies for incomes and health.  

Some of the options in the survey, such as VAT increases, were quite burdensome to model in the time available.  

Careful thought must be given to the definition of the base case the model is comparing against. 
\end{verbatim}

There are also interesting questions about how best to present results
of a model with such diverse outputs. For instance, since one is a stock
and the other a flow, can we count payments by a household from a wealth
tax in the same way as payments for income tax? And should we be
imputing a monetary value to any health improvements?

I have presented the TriplePC, a new microsimulation model with novel
but important features. We have established the importance and
practicality of using conjoint data to provide instant analysis of the
political implications of welfare packages, but also learned some
important lessons on how best to conduct integrated microsimulation and
conjoint analysis. We have also estimated new measures of the
relationship between income and health and shown how these, too, can be
integrated into the model. The TriplePC is available online at
https://triplepc.northumbria.ac.uk/ and its source code, linked on the
main site, is released under an open-source licence.

Conclusion

References

Buck, Alexy, and Graham Stark. 2001. Means Assessment: Options for
Change. LSRC Research Paper No.8. Legal Services Commission.

Buck, Alexy, and Graham Stark. 2003. Simplicity versus Fairness in Means
Testing: The Case of Civil Legal Aid. Fiscal Studies 24 (4): 427--49.
https://doi.org/10.1111/j.1475-5890.2003.tb00090.x.

Coulter, Fiona, Graham Stark, and Stephen Smith. 1995. `Micro-Simulation
Modelling of Personal Taxation and Social Security Benefits in the Czech
Republic'. IFS Working Paper Series W95/58.

Davis, Evan, Dilnot, Andrew S., Giles, Flanders Christopher, Johnson,
Paul, Ridge, Michael, Stark, Graham, Webb, Steven, and Whitehouse
Edward. 1992. Alternative proposals on tax and social security. London:
IFS.
https://ifs.org.uk/sites/default/files/output\_url\_files/comm29.pdf

Dilnot, Andrew, and Graham Stark. 1986a. The Poverty Trap, Tax Cuts, and
the Reform of Social Security. Fiscal Studies 7 (1): 1--10.
https://doi.org/10.1111/j.1475-5890.1986.tb00410.x.

Dilnot, Andrew, and Stark, Graham. 1986b. The Distributional
Consequences of Mrs Thatcher. Fiscal Studies 7 (2): 48--53.
https://doi.org/10.1111/j.1475-5890.1986.tb00421.x.

Dilnot, Andrew, Graham Stark, and Steven Webb. 1987. `The Targeting of
Benefits: Two Approaches'. Fiscal Studies 8 (1): 83--93.
https://doi.org/10.1111/j.1475-5890.1987.tb00434.x.

Dilnot, Andrew, Graham Stark, Ian Walker, and Steven Webb. 1987. `The
1987 Budget in Perspective'. Fiscal Studies 8 (2): 48--57.
https://doi.org/10.1111/j.1475-5890.1987.tb00535.x.

Duncan, Alan, and Graham Stark. 2000. `A Recursive Algorithm to Generate
Piecewise Linear Budget Contraints'. 2 May 2000.
https://doi.org/10.1920/wp.ifs.2000.0011.

Fry, Vanessa, and Graham Stark. 1987. `The Take-Up of Supplementary
Benefit: Gaps in the ``Safety Net''?' Fiscal Studies 8 (4): 1--14.
https://doi.org/10.1111/j.1475-5890.1987.tb00302.x.

Fry, Vanessa, and Graham Stark. 1992. The Takeup of Means-Tested
Benefits in the UK: The Transition to Income Support and Family Credit.
Institute for Fiscal Studies.

Fry, Vanessa, and Graham Stark. 1993. `The Take-up of Means-Tested
Benefits, 1984-90'. 1 January 1993.
https://doi.org/10.1920/re.ifs.1993.0041.

Giles, Christopher and McCrae, Julian. 1995. TAXBEN: the IFS
microsimulation tax and benefit model. Working Paper 1995.
https://ifs.org.uk/publications/taxben-ifs-microsimulation-tax-and-benefit-model

Johnson, Elliott A., Reed, Howard, Nettle, Daniel, Stark, Graham,
Chrisp, Joe, Howard, Neil, Gregory, Grace, Goodman, Cleopatra, Smith,
Matt, Coates, Jonathan, Robson, Ian, Parra-Mujica, Fiorella, Pickett,
Kate E. and Johnson, Matthew T. 2023. Treating Causes Not Symptoms:
Basic Income as a Public Health Measure, London: Compass,
\url{https://www.compassonline.org.uk/publications/treating-causes-not-symptoms-basic-income-as-a-public-health-measure}

Johnson, Elliott A., Reed, Howard, Stark, Graham, Villadsen, Aase,
Parra-Mujica, Fiorella, Kypridemos, Chris, Cookson, Richard, Nettle,
Daniel, Pickett, Kate E., Johnson, Matthew T. (2023) `The Health Case
for Basic Income: Case Study', UK Data Service, London: UKDS,
\url{https://ukdataservice.ac.uk/case-study/the-health-case-for-basic-income/}

Johnson, Paul, and Graham Stark. 1989. `Ten Years of Mrs Thatcher: The
Distributional Consequences'. Fiscal Studies 10 (2): 29--37.

Johnson, Paul, and Graham Stark. 1991. `The Effects of a Minimum Wage on
Family Incomes'. Fiscal Studies 12 (3): 88--93.
https://doi.org/10.1111/j.1475-5890.1991.tb00164.x.

Johnson, Paul, Steven Webb, and Graham Stark. 1990. `TAXBEN2: The New
IFS Tax and Benefit Model'. IFS Working Paper W90/5.
https://doi.org/10.1111/j.1475-5890.1989.tb00107.x.

Reed, H., Johnson, E. A., Stark, G., Nettle, D., Pickett, K. E. \&
Johnson, M. T. (2023) `Estimating the effects of Basic Income schemes on
mental and physical health among 18+ adults in the UK: a microsimulation
study', under review PLOS Mental Health.
https://doi.org/10.17605/OSF.IO/SKPYB

Reed, Howard, Nettle, Daniel, Parra-Mujica, Fiorella, Stark, Graham,
Wilkinson, Richard, Johnson, Matthew T., Johnson, Elliott. A. (2024)
`Examining the relationship between income and both mental and physical
health among adults in the UK: Analysis of 12 waves (2009-2022) of
Understanding Society', under review PLOS One. Working paper:
https://doi.org/10.17605/OSF.IO/SKPYB

Reed, Howard, and Graham Stark. 2009. Assessing the Ability to Pay for
the Fees Charged by Charities: Phase 1 \& 2. February, 36. Edinburgh:
Office of the Scottish Charities Regulator (OSCR).
http://www.oscr.org.uk/publications-and-guidance/affordability-reportphase-2/.

Reed, Howard, and Graham Stark. 2011a. Modelling the Costs for
Individuals and Public Authorities in Wales of Alternative Funding
Systems for the Long-Term Care of Adults: Stage 1 Report: Building a
Forecasting Model for Long-Term Care in Wales. Welsh Assembly
Government.

Reed, Howard, and Graham Stark. 2011b. Modelling the Costs for
Individuals and Public Authorities in Wales of Alternative Funding
Systems for the Long-Term Care of Adults. Welsh Assembly Government.

Reed, Howard, and Graham Stark. 2013. Costing the ``When I Am Ready''
Scheme. Action for Children Wales/Gweithredu dros Blant.

Reed, Howard, and Graham Stark. 2018. Tackling Child Poverty Delivery
Plan: Forecasting Child Poverty in Scotland. Scottish Government.
http://www.gov.scot/Publications/2018/03/2911/0.

Reed, Howard, and Graham Stark. 2020. Giving Care Leavers the Chance to
Stay: Staying Put Six Years on: Technical Report. Action for Children
England. https://doi.org/10.1111/j.1475-5890.1988.tb00319.x.

Robinson, Bill, and Graham Stark. 1988. `The Tax Treatment of Marriage:
What Has the Chancellor Really Achieved?' Fiscal Studies 9 (2): 48--56.

Stark, Graham. 1988. `Partially Transferable Allowances'. Fiscal Studies
9 (1): 29--40. https://doi.org/10.1111/j.1475-5890.1988.tb00310.x.

Stark, Graham. 2021. Staying Put Six Years on: 2021 Update. Action for
Children England.

Stark, Graham, Johnson, Elliott A., Reed, Howard, Nettle, Daniel, and
Johnson, Matthew T. 2024 in press. The Public Policy Preference
Calculator (TriplePC): Developing a comprehensive welfare policy
microsimulation. International Journal of Microsimulation. Working
paper: https://doi.org/10.17605/OSF.IO/SKPYB



\end{document}
